\newpage
\section{Summary}
The Gaia ultra-cool dwarf sample (GUCDS) is a list of objects with
published spectral types later than M7. The basic philosophy and first
effort is discussed in \cite{2017MNRAS.469..401S}. A number of these
UCDs have been found to be in binary systems with Gaia objects and here we
reproduced basic information for all of the found systems todate.


To generate the list of benchmark systems for each UCD we searched for
candidate companions in the Gaia EDR3 that met the following criteria:
\begin{equation} \label{eq1}
\begin{split}
\rho         & < 100 \varpi, \\
\Delta\varpi & < {\rm max}[1.0,3\sigma_{\varpi}],\\
\Delta\mu    & < 0.1\mu,\\
\Delta\theta & < 15\degree
\end{split}
\end{equation}

\noindent where $\rho$ is the separation on the sky in arcseconds,
$\Delta\varpi$ is the difference of the GUCDS and candidate primary
parallaxes, $\varpi$ and $\sigma_{\varpi}$ are the parallax and error
of the GUCDS object- when no parallax has been measured we used
spectro-photometric distances. $\Delta\mu$ is the difference of the
total proper motions, and $\Delta\theta$ is the difference of the
proper motion position angles. The chosen $\rho$ criterion is
equivalent to 100,000\,au, which is a conservative upper limit for a
projected physical separation ($s$). This will meet the binding energy
criterion of $ |U_g^*| = G M_1 M_2 / s > 10^{33} J $ as developed by
\cite{2009A&A...507..251C} for a 0.1~M$_{\odot}$ + 2~M$_{\odot}$
system \cite{2010AJ....139.2566D}. The $\Delta\varpi$
criterion is based on a consideration of the errors, standard
3$\sigma$ criterion or 1.0\,mas, to allow for solutions that had
unrealistically low errors. For the modulus and position angles of the
proper motion, criteria based on the errors would remove nearby
objects with significant orbital motion, hence we simply choose hard
criteria of $\sim$10\% in both parameters. This is large enough to
accommodate most orbital motion, but small enough to avoid most false
positives.

Each target is provided with basic information made from the original
lists, Gaia catalogs, online sources and a latex commetn file. The
main latex files are generated automatically but comments can be added
to a particular target by editing the corresponding latex comment file
which is conserved between successive latex generations.



